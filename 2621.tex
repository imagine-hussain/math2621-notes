%%%%%%%%%%%%%%%%%%%%%%%%%%%%%%%%%%%%%%%%%%%%%%%%%%%%%%%%%%%%%%%%%%%%%%%%%%
%%%%%%%%%%%%%%%%%%%%%%%%%%%%%%%%%%%%%%%%%%%%%%%%%%%%%%%%%%%%%%%%%%%%%%%%%%
%%%%%%%%%%%%%%%%%%%   Package Preamble  %%%%%%%%%%%%%%%%%%%%%%%%%%%%%%%%%%
%%%%%%%%%%%%%%%%%%%%%%%%%%%%%%%%%%%%%%%%%%%%%%%%%%%%%%%%%%%%%%%%%%%%%%%%%%
%%%%%%%%%%%%%%%%%%%%%%%%%%%%%%%%%%%%%%%%%%%%%%%%%%%%%%%%%%%%%%%%%%%%%%%%%%

\documentclass[12pt, letterpaper]{article}
\usepackage[margin=1in]{geometry}
\usepackage{amsmath}
\usepackage{amssymb}
\usepackage{mathtools}
\usepackage{fancyhdr}
\usepackage[utf8]{inputenc}
\usepackage{dirtytalk}                      % \say command for quotes
\usepackage{ wasysym }
\usepackage{graphicx}

\usepackage[pdftitle=Higher Complex Analysis]{hyperref}
\hypersetup {                                % Formatting for hyperlinks
    colorlinks,
    citecolor=black,
    filecolor=black,
    linkcolor=black,
    urlcolor=black
    }
    
\graphicspath{}								% Path for images

\pagestyle{fancy}
\setlength{\headheight}{15pt}
\renewcommand{\headrulewidth}{0pt}
\renewcommand{\footrulewidth}{0pt}	
\title{Higher Complex Analysis}
\author{Hussain Nawaz \\ hussain.nwz000@gmail.com}
\date{2021 T3}

\rhead{}
\lhead{}

%%%%%%%%%%%%%%%%%%%%%%%%%%%%%%%%%%%%%%%%%%%%%%%%%%%%%%%%%%%%%%%%%%%%%%%%%%
%%%%%%%%%%%%%%%%%%%%%%%%%%%%%%%%%%%%%%%%%%%%%%%%%%%%%%%%%%%%%%%%%%%%%%%%%%
%%%%%%%%%%%%%%%%%%%    Document Contents    %%%%%%%%%%%%%%%%%%%%%%%%%%%%%%
%%%%%%%%%%%%%%%%%%%%%%%%%%%%%%%%%%%%%%%%%%%%%%%%%%%%%%%%%%%%%%%%%%%%%%%%%%
%%%%%%%%%%%%%%%%%%%%%%%%%%%%%%%%%%%%%%%%%%%%%%%%%%%%%%%%%%%%%%%%%%%%%%%%%%

\begin{document}
\maketitle
\tableofcontents
\newpage

    \section{Assumed Knowledge}
    \begin{itemize}
        \item Arithmetic of complex numbers
        \item Cartesian and Polar representation
        \item Argand Diagram
        \item de Moivre's Theorem
        \item Extracting \(n\)-th roots
    \end{itemize}

%%%%%%%%%%%%%%%%%%%%%%%%%%%%%%%%%%%%%%%%%%%%%%%%%%%%%%%%%%%%%%%%%%%%%%%%%%
%%%%%%%%%%%%%%%%%%%%%%%%%%%%%%%%%%%%%%%%%%%%%%%%%%%%%%%%%%%%%%%%%%%%%%%%%%
%%%%%%%%%%%%%%%%%%%%%%%%%%%%%%%%%%%%%%%%%%%%%%%%%%%%%%%%%%%%%%%%%%%%%%%%%%

    \section{Inequalities and Sets}
    \subsection{Triangle Inequalities}
    \paragraph{The triangle inequality} For all \(w, z\in \mathbb{C}\),
    \[ |w + z| \leq |w| + |z|.\]
    This can be generalised for an arbitrary amount of complex numbers such that
    \[|z_0 + z_1 + z_2 + \dots z_n| \leq |z_0| + |z_1| + |z_2| + \dots |z_n|.\]

    \paragraph{Reverse Triangle Inequality}
    For all \(w, z\in \mathbb{C}\),
    \[||w|-|z|| \leq |w-z|.\]
   
    \paragraph{Lemma} For all real \(\theta\),
    \[|e^{i\theta} - 1 | \leq |\theta|.\]

    \subsection{Sets}
    \paragraph{Open Ball}
    The open ball with center \(z_0\) and radius \(\epsilon\), \(\mathrm{B}(z_0, \epsilon\) is the set,
    \[\{z\in \mathbb{C}: |z-z_0| < \epsilon\}.\]
    
    \paragraph{Types of Points} For any point \(z\) in \(mathbb{C}\), there are 3 mutually exclusive and exhaustive possibilities:
    \begin{itemize}
        \item Interior Point,
        \item Boundary Point,
        \item Exterior Point.
    \end{itemize}
    
    \paragraph{Types of Sets}
    \begin{itemize}
        \item A set is \textbf{open} if all of its points are interior points.
        \item A set is \textbf{closed} if it contains all of its boundary points or, its complement is open.
        \item The \textbf{closure} of a set \(S\) written as \(\bar{S}\) is the set plus its boundary points.
        \item A set \(S\) is \textbf{bounded} if \(S\subseteq B(0, R)\) for some real \(R\).
        \item A set \(S\) is \textbf{compact} if it is both closed and bounded.
        \item A set is a \textbf{region} if it is an open set, with some, none, or all of its boundary points.
    \end{itemize}

    \paragraph{Paths}
    A polygonal path is a sequence of line segments such that the end point of one line is the start point of the next line.
    A \underline{simple, closed} path is one that does not cross itself and, has the feature that its final point is the same as its initial point.
    The complement of a simple closed path has two pieces. A bounded interior and unbounded exterior.
    \begin{itemize}
        \item A set is \textbf{path connected} if and two points in the set can by joined by a path within that set.
        \item A set is \textbf{simply connected} if it is connected and the interior of every connected path remains entirely within the set. In other words, the set has no holes.
        \item The set is a \textbf{domain} if it is open and path connected.
    \end{itemize} 

%%%%%%%%%%%%%%%%%%%%%%%%%%%%%%%%%%%%%%%%%%%%%%%%%%%%%%%%%%%%%%%%%%%%%%%%%%
%%%%%%%%%%%%%%%%%%%%%%%%%%%%%%%%%%%%%%%%%%%%%%%%%%%%%%%%%%%%%%%%%%%%%%%%%%
%%%%%%%%%%%%%%%%%%%%%%%%%%%%%%%%%%%%%%%%%%%%%%%%%%%%%%%%%%%%%%%%%%%%%%%%%%

    \section{Functions of a Complex Variable}

    Complex functions may be written as 
    \[f(z) = f(x,y) = u(x, y) + iv(x, y).\]

    \paragraph{The Fundamental Theorem Of Algebra}
    Every non-constant complex polynomial of degree \(d\) factorizes uniquely. That is, there exist \(\alpha_1, \alpha_2 \dots alpha_d\) such that 
    \[p(z) = c\prod_{j=1}^{d} (z-\alpha_j).\]

%%%%%%%%%%%%%%%%%%%%%%%%%%%%%%%%%%%%%%%%%%%%%%%%%%%%%%%%%%%%%%%%%%%%%%%%%%
%%%%%%%%%%%%%%%%%%%%%%%%%%%%%%%%%%%%%%%%%%%%%%%%%%%%%%%%%%%%%%%%%%%%%%%%%%
%%%%%%%%%%%%%%%%%%%%%%%%%%%%%%%%%%%%%%%%%%%%%%%%%%%%%%%%%%%%%%%%%%%%%%%%%%

    \section{Fractional Linear Transformations}
    \subsection{Introduction to FLTs}
    
    Fractional Linear Transformations (FLTs) are functions of the form
    \[f(z) \frac{az+b}{cz+d}, \quad\quad a,b,c,d\in\mathbb{C}, \quad\quad ad-bc\neq 0.\]

    They may be represented as a complex matrix \(M\) such that 
    \[M = \begin{pmatrix} a & b \\ c & d \end{pmatrix}.\]

    If the determinant of this matrix is zero, then the function is constant.

    \paragraph{Factorising Matrices} Every \(2\times 2\) matrix may be written as a product of at most \(3\) matrices of the following two forms:
    \[
        \begin{pmatrix}
            a & b \\ 0 & d
        \end{pmatrix} 
        \quad\quad \text{and} \quad \quad
        \begin{pmatrix}
            0 & 1 \\ -1 & 0
        \end{pmatrix}.
    \]
    
    \paragraph{Inverse of FLT}
    Suppose that \[ w = \frac{az + b}{cz + d}.\]
    Then, \[ z = \frac{dw - b}{a-cw}.\]
    
    \subsection{Images of FLTs}
    \paragraph{Image Under a Line or Circle}
    Let \(T_M\) be a fractional linear transformation. Then, the image of a line or a circle under \(T_M\) is either a line or a circle.
    If \(\frac{-d}{c}\in \text{Domain}(T_M)\) then, the image of a FLT is a line.
    Otherwise, the image is bounded and is thus a circle.
    
%%%%%%%%%%%%%%%%%%%%%%%%%%%%%%%%%%%%%%%%%%%%%%%%%%%%%%%%%%%%%%%%%%%%%%%%%%
%%%%%%%%%%%%%%%%%%%%%%%%%%%%%%%%%%%%%%%%%%%%%%%%%%%%%%%%%%%%%%%%%%%%%%%%%%
%%%%%%%%%%%%%%%%%%%%%%%%%%%%%%%%%%%%%%%%%%%%%%%%%%%%%%%%%%%%%%%%%%%%%%%%%%
    
    \section{Limits and Continuity}
    \subsection{Limits}
    Most rules from real numbers apply.
    \paragraph{Uniqueness of Limit}
    Suppose that \(f\) is a complex function and that \(z_0\in \text{Domain}(f^{-1}.\)
    If \(\lim_{z\to z_0}\) exists, then it is unique.
    
    \subsection{Continuity}
    \paragraph{Definition}
    Suppose that \(f\) is a complex function.
    \(f\) is continuous at the point \(z_0\) if \(f(z_0)\) is defined and
    \(\lim_{z\to z_0} = f(z_0)\). The function if it is continuous on a set \(S\) is it is continuous at all points on its domain.
    
    \paragraph{Some Continous Functions}
    The following functions are continuous for all \(z\in \mathbb{C}\) where \(z\neq 0\).
    \begin{itemize}
        \item \(z\mapsto z\)
        \item \(z\mapsto |z|\)
        \item \(z\mapsto \Re(z)\)
        \item \(z\mapsto \Im(z)\)
    \end{itemize}
    
    \paragraph{Composition of Continuous Functions} If \(c\int\mathbb{C}\), \(f, g\) are continuous functions \(S\mapsto \mathbb{C}\) then, 
    \[cd, f + g, |f|, f, \Re{f}, \Im{f}\]
    are continuous on \(S\) as well as \(\frac{f}{g}\) where \(g(z) \neq 0.\)

%%%%%%%%%%%%%%%%%%%%%%%%%%%%%%%%%%%%%%%%%%%%%%%%%%%%%%%%%%%%%%%%%%%%%%%%%%
%%%%%%%%%%%%%%%%%%%%%%%%%%%%%%%%%%%%%%%%%%%%%%%%%%%%%%%%%%%%%%%%%%%%%%%%%%
%%%%%%%%%%%%%%%%%%%%%%%%%%%%%%%%%%%%%%%%%%%%%%%%%%%%%%%%%%%%%%%%%%%%%%%%%%

    \section{Complex Differentiability}
    Suppose that \(S \subseteq\mathbb{C}\) and that \(f : S \to C\) is a complex function.
    Then we say that f is differentiable at the point \(z_0\) in \(S\) if
    \[ 
        \lim_{z\to z_0} \frac{f(z) - f(z_0)}{z-z_0}\]
    exists and is finite. If so, then it is the derivative of \(f\) at \(z_0\).

    \subsection{Cauchy-Reimann Equations}
    If \(f\) is differentiable at every point of an open set, then the Cauchy-Reimann equations hold at every point of that set.
    \paragraph{The equations}
    \[
        \frac{\partial u}{\partial x} = \frac{\partial v}{\partial y}
        \quad \quad \text{and} \quad \quad
        \frac{\partial u}{\partial y} = -\frac{\partial v}{\partial x}.
    \]

    \paragraph{The derivative in terms of \(u\) and \(v\)}
    \[
        f'(z) = u_x + i v_x = v_y  - i u_y.
    \]
    
    \paragraph{Holomorphic Functions}
    Suppose that \(\Omega\) is an open subset of \(\mathbb{C}\) and 
    \(f: \Omega \to \mathbb{C}\) is a function. If \(f\) is differentiable
    at every point of \(\Omega\) then, we say that \(f\) is \textbf{holomorphic}
    or \textbf{complex analytic} or \textbf{analytic} in \(\Omega\).
    If \(f\) is analytic on the entire complex plane then, it is \textbf{entire.}

%%%%%%%%%%%%%%%%%%%%%%%%%%%%%%%%%%%%%%%%%%%%%%%%%%%%%%%%%%%%%%%%%%%%%%%%%% 
%%%%%%%%%%%%%%%%%%%%%%%%%%%%%%%%%%%%%%%%%%%%%%%%%%%%%%%%%%%%%%%%%%%%%%%%%% 
%%%%%%%%%%%%%%%%%%%%%%%%%%%%%%%%%%%%%%%%%%%%%%%%%%%%%%%%%%%%%%%%%%%%%%%%%% 

    \section{Harmonic Functions}
    \paragraph{Definition}
    Suppose that \(u: \Omega \to \mathbb{R}\) is a function where \(\Omega\)
    is an open subset of \(\mathbb{R}^2\) and, that \(u\) is twice continuously 
    differentiable. \(u\) is harmonic if, it satisfies \textit{Laplace's Equation}:    
    \[
        \frac{\partial^2 u}{\partial x^2} +  \frac{\partial^2 u}{\partial y^2}
        = 0.
    \]

    \paragraph{Harmonic and Holomorphic Functions} To find a harmonic function,
    it is sufficient to find a holomorphic function.
    Suppose that \(f\int H(\Omega)\) where \(\Omega\) is an open subset of 
    \(\mathbb{C}\), that \(f\) is twice continuously differentiable and that
    \[f(x + iy) = u(x, y) + iv(x, y)\]
    for all \(x+iy\in\Omega\) where \(u\) and \(v\) are real-valued.
    Then \(u\) and \(v\) are harmonic functions.

    \paragraph{Infinite Differentiability}
    For the above, the twice differentiable requirement is redundant since, if 
    \(f\in H(\Omega) then,\) \(f\) is infinitely differentiable on \(\Omega\).
    
    \paragraph{Harmonic Conjugate}
    If \(\Omega\) is a simply connected domain, and \(u: \Omega \to \mathbb{R}\)
    is harmonic then, there exists a harmonic function \(v: \Omega \to \mathbb{R}\)
    such that \(f\) given by 
    \[
        f(x + iy) = u(x,y) + iv(x,y)
    \] in \(\Omega\) is holomorphic. In fact, any such functions \(v\)
    differ only by a complex constant.
    This function \(v\) is called the \textbf{harmonic conjugate} of \(u\)

    \paragraph{Function From Harmonic Components}
    If \(u, v\) is found then, you may evaluate \(f(z, 0)\) in place of 
    \(f(x, y)\) and allow that to be your \(f(z)\) due to analytic continuity.
    See more about this later.
    
%%%%%%%%%%%%%%%%%%%%%%%%%%%%%%%%%%%%%%%%%%%%%%%%%%%%%%%%%%%%%%%%%%%%%%%%%%
%%%%%%%%%%%%%%%%%%%%%%%%%%%%%%%%%%%%%%%%%%%%%%%%%%%%%%%%%%%%%%%%%%%%%%%%%%
%%%%%%%%%%%%%%%%%%%%%%%%%%%%%%%%%%%%%%%%%%%%%%%%%%%%%%%%%%%%%%%%%%%%%%%%%%
%%%%%%%%%%%%%%%%%%%%%%%%%%%%%%%%%%%%%%%%%%%%%%%%%%%%%%%%%%%%%%%%%%%%%%%%%%
    
    \section{Power Series}
    \subsection{Definition and Convergence}
    
    \paragraph{Definition}
    A complex power series is an expression of the form
    \[\sum_{n=0}^{\infty} a_n \left( z-z_0\right),\]
    centered at \(z_0\) where, \(a_n\) are all fixed complex numbers.

    \paragraph{Ratio Test}
    The radius of convergence \(p\), may be found as 
    \[p = \lim_{n\to\infty} \frac{|a_n|}{a_{n+1}}\]
    as long as the limit exists or, is \(+\infty\).
    
    \paragraph{Root Test} The radius of convergence \(p\) is given by
    \[p = \lim_{n\to\infty} \frac{1}{|a_n|^\frac{1}{n}}\]
    as long as the limit exists or, is \(+\infty\).
    \newline\indent
    In other to use the root test for coefficients with factorials,
    \textbf{Stirling'g formula} may be useful.
    \[
        \lim_{n\to\infty} \frac{1}{\ln((2\pi)^1/2 e^{-n} n^{n+1/2})} = 1
        \text{ and }
        \lim_{n\to\infty} \frac{n!}{(2\pi)^{1/2} e^{-n} n^{n+1/2}} = 1.
    \] 
   
    \subsection{Algebra and Calculus of Power Series}
    Suppose that both power series
    \[f(z) = \sum_{n=0}^\infty a_n (z-z_0)^n
    \quad \text{and} \quad g(z) = \sum_{n=0}^\infty b_n (z-z_0)^n \]
    converge in \(B(z_0, p\). Then, the following series also converge in 
    \(B(z_0, p)\):
    \begin{enumerate}
        \item \(\sum ca_n(z-z_0)^ n= c\sum a_n(z-z_0)^n\)
        \item \(\sum (a_n + b_n)(z-z_0)^n = \sum a_n(z-z_0)^n + \sum b_n(z-z_0)^n\)
        \item \(\sum_{n=0}^{\infty} a_n(z-z_0)^n \times \sum b_n(z-z_0)^n
        = \sum_{n=0}^{\infty} c_n (z-z_0)^n\) where
        \(c_n = \sum_{j=0}^{n} a_{j} b_{n-j}\)
    \end{enumerate}

    \paragraph{Derivatives of Power Series}
    The derivative of the power series can be found by simply taking the 
    sum of derivatives of each element. The derivative exists in the same
    radius of convergence as the original function.
    That is, for \(f(z) = \sum_{n=0}^{\infty} a_n (z-z_0)^n\),
    \[f'(z) = \sum_{n=0}^{\infty}a_n n (z-z_0)^{n-1}.\]
    An infinite power series of the form above may be differentiated infinitely.
    
%%%%%%%%%%%%%%%%%%%%%%%%%%%%%%%%%%%%%%%%%%%%%%%%%%%%%%%%%%%%%%%%%%%%%%%%%%
%%%%%%%%%%%%%%%%%%%%%%%%%%%%%%%%%%%%%%%%%%%%%%%%%%%%%%%%%%%%%%%%%%%%%%%%%%
%%%%%%%%%%%%%%%%%%%%%%%%%%%%%%%%%%%%%%%%%%%%%%%%%%%%%%%%%%%%%%%%%%%%%%%%%%

    \section{Exponential, Hyperbolic and Trigonometric Function}
    \subsection{The exponential function}

    \paragraph{Definition}
    We may define the exponential function with the following power series:
    \[\exp(z) = \sum_{n=0}^{\infty} \frac{z^n}{n!} \quad\quad
    \forall z\in\mathbb{C}.\]
    If a function satisfies \(f(0) = 1\) and \(f'(z) = f(z)\) then, that 
    function must be the exponential function.

    \subsection{The hyperbolic functions}
    We may define the hyperbolic functions with the following power series:
    \[
        \cosh(z) = \frac{e^z + e^{-z}}{2} = \sum_{n=0}^{\infty} \frac{z^{2n}}{(2n)!}
    \]
    and 
    \[
        \sinh(z) = \frac{e^z - e^{-z}}{2}
        = \sum_{n=0}^{\infty} \frac{z^{2n+1}}{(2n+1)!}.
    \]
    Observe that
    \begin{itemize}
        \item \(\cos(iz) = \cosh(z)\)
        \item \(\cosh(iz) = \cos(z)\)
        \item \(\sin(iz) = i\sinh(z)\)
        \item \(\sinh(iz) = i\sin(z)\)
    \end{itemize}

    \subsection{The trigonometric functions}
    We may define the trigonometric functions with the following power series:
    \[
        \cos(z) = \frac{e^z + e^{-z}}{2} = \sum_{n=0}^{\infty} 
        \frac{(-1)^n z^{2n}}{(2n+1)!}
    \]
    and,
    \[ 
        \sin(z) = \frac{e^z - e^{-z}}{2} = \sum_{n=0}^{\infty}
        \frac{(-1)^n z^{2n}}{(2n+1)!}.
    \]

%%%%%%%%%%%%%%%%%%%%%%%%%%%%%%%%%%%%%%%%%%%%%%%%%%%%%%%%%%%%%%%%%%%%%%%%%% 
%%%%%%%%%%%%%%%%%%%%%%%%%%%%%%%%%%%%%%%%%%%%%%%%%%%%%%%%%%%%%%%%%%%%%%%%%% 
%%%%%%%%%%%%%%%%%%%%%%%%%%%%%%%%%%%%%%%%%%%%%%%%%%%%%%%%%%%%%%%%%%%%%%%%%% 

    \section{Logarithms and Roots}
    \paragraph{Square Root}
    We define the principle value of the square root as:
    \[
        PV w^{\frac{1}{2}} = \begin{cases}
            |w|^\frac{1}{2} e^{i\mathrm{Arg}(w)/2} & \text{if \(w \neq 0\)} \\
            0   & \text{if \(w = 0\).}
        \end{cases}
    \]
    
    \paragraph{n-th Roots}
    The principle value of the \(n\)th root is given by 
    \[
        PV z^{1/n} = \exp\left(\frac{\mathrm{Log}(z)}{n}\right)
        = |z|^\frac{1}{n} e^{\frac{i\mathrm{Arg(z)}}{n}}.
    \]
    The function \(PVz^{1/n}\) is differentiable in
    \(\mathbb{C} \backslash (-\infty, 0].\)

%%%%%%%%%%%%%%%%%%%%%%%%%%%%%%%%%%%%%%%%%%%%%%%%%%%%%%%%%%%%%%%%%%%%%%%%%%
%%%%%%%%%%%%%%%%%%%%%%%%%%%%%%%%%%%%%%%%%%%%%%%%%%%%%%%%%%%%%%%%%%%%%%%%%%
%%%%%%%%%%%%%%%%%%%%%%%%%%%%%%%%%%%%%%%%%%%%%%%%%%%%%%%%%%%%%%%%%%%%%%%%%%

    \section{Inverses of exponential and related function}

    \paragraph{Logarithm}
    We define the complex logarithm as 
    \[\log(z) = \ln(z) + \arg(z)i.\]
    
    \paragraph{Complex Powers}
    We may define raising to a complex power \(\alpha\) as the following
    \[ z^\alpha = \exp(\alpha \log (z).\]
    That is,
    \[z^\alpha = \exp(\alpha \ln|z| + i\mathrm{Arg}(z)\alpha + \alpha 2k\pi i,\]
    where, \(k\) is an integer.
    \newline \indent
    The function \(z\mapsto PVz^\alpha\) is differentiable on 
    \(\mathbb{C} \backslash (-\infty, 0).\)
    
    \paragraph{Inverse sinh}
    We define the principal inverse of \(\sinh\) as 
    \[PV \sinh^{-1} w = \mathrm{Log}(w + PV(w^2 + 1)^{1/2}).\]
    \paragraph{Inverse cosh}
    We define the principal inverse of \(\sinh\) as 
    \[PV \cosh^{-1} w = \mathrm{Log}(w + PV(w + 1)^{1/2} PV (w-1)^{1/2}).\]

%%%%%%%%%%%%%%%%%%%%%%%%%%%%%%%%%%%%%%%%%%%%%%%%%%%%%%%%%%%%%%%%%%%%%%%%%%
%%%%%%%%%%%%%%%%%%%%%%%%%%%%%%%%%%%%%%%%%%%%%%%%%%%%%%%%%%%%%%%%%%%%%%%%%%
%%%%%%%%%%%%%%%%%%%%%%%%%%%%%%%%%%%%%%%%%%%%%%%%%%%%%%%%%%%%%%%%%%%%%%%%%%
    
    \section{Contour Integrals}
    \paragraph{Definition}  A contour is an oriented range of a piece-wise
    smooth curve in the complex plane.

    \paragraph{Contour Integrals} Given a piecewise smooth curve
    \(\gamma(t): [a, b] \to \mathbb{C}\) and a continuous (not necessarily differentiable)
    function \(f\) defined on the range of \(\gamma\), we define the complex
    line integral of \(f\) over \(gamma\) as
    \[\int_{\gamma} f(z) dz = \int_{a}^{b} f(\gamma(t)) \gamma '(t)dt. \]

    \paragraph{Properties}
    \begin{itemize}
        \item The integral is linear
        \item The integral is independent of the parametrization
        \item The integral is additive for joins
        \item The integral is dependent of the orientation
        \item The size of the integral depends on the size of the curve and function. See the ML lemma
    \end{itemize}  

    \paragraph{The ML Lemma}
    \[|\int_\gamma f(z) dz \leq ML\]
    where \(L\) is the length of \(\gamma\) and \(M\) is the maximizer.
    That is, \(M\) is some number such that, for all \(z\in \mathrm{range}(\gamma),\) 
    \(f(z) \leq M\).
    
%%%%%%%%%%%%%%%%%%%%%%%%%%%%%%%%%%%%%%%%%%%%%%%%%%%%%%%%%%%%%%%%%%%%%%%%%%
%%%%%%%%%%%%%%%%%%%%%%%%%%%%%%%%%%%%%%%%%%%%%%%%%%%%%%%%%%%%%%%%%%%%%%%%%%
%%%%%%%%%%%%%%%%%%%%%%%%%%%%%%%%%%%%%%%%%%%%%%%%%%%%%%%%%%%%%%%%%%%%%%%%%%
    
    \section{The Cauchy Goursat Theorem}

    \paragraph{The Cauchy Goursat Theorem}
    Suppose that \(\Omega\) is a simply connected domain such that
    \(f\in H(\Omega)\) and that \(\Gamma\) is a closed contour in \(\Omega\).
    Then,
    \[\int_\Gamma f(z) dz = 0.\]
    
    \paragraph{Existence Of Primitives}
    Suppose that \(\Omega\) is a simply connected domain in \(\mathbb{C}\)
    and that \(f\in H(\Omega)\). Then, there exists a function \(F\) on
    \(\Omega\) such that 
    \[\int_\Gamma f(z) = F(q) - F(p)\]
    for all contours in \(\Omega\) from \(p\) to \(q\).
    Further, it is the case that \(F\) is differentiable and, \(F' = f\).
 
%%%%%%%%%%%%%%%%%%%%%%%%%%%%%%%%%%%%%%%%%%%%%%%%%%%%%%%%%%%%%%%%%%%%%%%%%%    
%%%%%%%%%%%%%%%%%%%%%%%%%%%%%%%%%%%%%%%%%%%%%%%%%%%%%%%%%%%%%%%%%%%%%%%%%%    
%%%%%%%%%%%%%%%%%%%%%%%%%%%%%%%%%%%%%%%%%%%%%%%%%%%%%%%%%%%%%%%%%%%%%%%%%%    

    \section{Cauchy's Integral Formula}

    \subsection{The integral formula}
    
    Suppose that \(\Omega\) is a simply connected domain in \(\mathbb{C}\),
    that \(f\) is holomorphic on \(\Omega\), that \(\Gamma\) is a simply
    connected domain and that \(w \in \mathrm{Int}(\Gamma)\).
    Then,
    \[ f(w) = \frac{1}{2\pi i} \int_\Gamma \frac{f(z)}{z-w}dz.\]
    In other words, 
    \[ \int_\Gamma \frac{f(z)}{z-w} = 2\pi i f(w).\]


    \paragraph{Cauchy's Integral Formula with Power Series}
    With the same conditions as above, where \(\Omega = B(z_0, R)\).
    \[f(w) = \sum_{n=0}^{\infty} c_n (z-z_0)^n\]
    where,
    \[c_n = \frac{1}{2\pi i} \int_\Gamma \frac{f(z)}{(z-z_0)^n}dz \]
    The radius of convergence is at least \(R\).

    We may use this, combined with the fact that \(f^{(n)} (z_0) = n!c_n\)
    to get Cauchy's Generalised Integral Formula.

    \subsection{The Generalised Integral Formula}
    
    \paragraph{Cauchy's Generalised Integral Formula}
    \[ f^{(n)} (z_0) = \frac{n!}{p\pi i} \int_\Gamma \frac{f(z)}{(z-z_0)^{n+1}}.\]

    \paragraph{Liouville's Theorem} Suppose that \(f\) is a bounded entire  
    function. Then, \(f\) is constant.

%%%%%%%%%%%%%%%%%%%%%%%%%%%%%%%%%%%%%%%%%%%%%%%%%%%%%%%%%%%%%%%%%%%%%%%%%%
%%%%%%%%%%%%%%%%%%%%%%%%%%%%%%%%%%%%%%%%%%%%%%%%%%%%%%%%%%%%%%%%%%%%%%%%%%
%%%%%%%%%%%%%%%%%%%%%%%%%%%%%%%%%%%%%%%%%%%%%%%%%%%%%%%%%%%%%%%%%%%%%%%%%%

    \section{Morera's Theorem and Analytic Continuation}

    \paragraph{Morera's Theorem}
    Suppose that \(\Omega\) is a domain and that the function
    \(f: \Omega \to \mathbb{C}\) is continuous and that 
    \[ \int_\Gamma f(z) dz = 0\]
    whenever \(\Gamma\) is a closed contour in \(\Omega.\) Then,
    \(f\) is holomorphic in \(Omega\).

    \paragraph{Analytic Continuation}
    Suppose that \(\Upsilon\) is a nonempty, open subset of a domain 
    \(\Omega\) in \(\mathbb{C}\) and that \(f\in H(\Omega)\).
    If \(f(z) = 0\) for all \(z\) in \(\Upsilon\) then, \(f(z) = 0\) for 
    all \(z\) in \(\Omega\).

%%%%%%%%%%%%%%%%%%%%%%%%%%%%%%%%%%%%%%%%%%%%%%%%%%%%%%%%%%%%%%%%%%%%%%%%%%
%%%%%%%%%%%%%%%%%%%%%%%%%%%%%%%%%%%%%%%%%%%%%%%%%%%%%%%%%%%%%%%%%%%%%%%%%%
%%%%%%%%%%%%%%%%%%%%%%%%%%%%%%%%%%%%%%%%%%%%%%%%%%%%%%%%%%%%%%%%%%%%%%%%%%

    \section{Taylor Series}
    A taylor series for the function \(f\) around \(z_0\) is of the form
    \[\sum_{n=0}^{\infty} \frac{f^{(n)}(z_0)}{n!} (z-z_0)^n.\]
    
    \paragraph{Algebra and Calculus of Taylor Series}
    The same rules that apply for power series, also apply for Taylor series.

    \paragraph{Differentiability}
    Suppose that the taylor series for \(f\) converges in a ball centered at
    \(z_0\), of radius\(r\). Then \(f\) is differentiable on that same
    domain.

    \paragraph{Lagrange inversion theorem}
    Suppose that \(f\) is holomorphic on \(\Omega\) and that 
    \(f(a) = b\) and \(f'(a) \neq 0\).
    Then, there is a holomorphic function \(g\) defined on an open set
    \(\Upsilon\) containing \(b\) such that \(g\circ f(z) = z\) for all
    \(z\) near \(a\) and, \(f\circ g(w) = w\) for all \(w\) near b.
    Further, \(g(w) = a + \sum_{n=1}^{\infty} c_n (w-b)^n \) where
    \[c_n = \frac{1}{n!} \lim_{z\to a} \frac{d^{n-1}}{dz^{n-1}}\]
    when \(n\geq 1\).

%%%%%%%%%%%%%%%%%%%%%%%%%%%%%%%%%%%%%%%%%%%%%%%%%%%%%%%%%%%%%%%%%%%%%%%%%%
%%%%%%%%%%%%%%%%%%%%%%%%%%%%%%%%%%%%%%%%%%%%%%%%%%%%%%%%%%%%%%%%%%%%%%%%%%
%%%%%%%%%%%%%%%%%%%%%%%%%%%%%%%%%%%%%%%%%%%%%%%%%%%%%%%%%%%%%%%%%%%%%%%%%%

    \section{Laurent's Theorem}
    Suppose that \(A\) is the annulus
    \(\{z\in \mathbb{C} : R_1 < |z-z_0| < R_2\}\) and
    \(R_1 < r < R_2\). If \(f\) is holomorphic on \(A\) then,
    \[ f(w) = \sum_{n=-\infty}^{\infty} c_n (w-z_0)^n
    \quad \forall w\in A\] where
    \[c_n = \frac{1}{2 \pi i} \int_{\partial B(z_0, r)}
    \frac{f(z)}{(z-z_0)^{n+1}}.\]

%%%%%%%%%%%%%%%%%%%%%%%%%%%%%%%%%%%%%%%%%%%%%%%%%%%%%%%%%%%%%%%%%%%%%%%%%%
%%%%%%%%%%%%%%%%%%%%%%%%%%%%%%%%%%%%%%%%%%%%%%%%%%%%%%%%%%%%%%%%%%%%%%%%%%
%%%%%%%%%%%%%%%%%%%%%%%%%%%%%%%%%%%%%%%%%%%%%%%%%%%%%%%%%%%%%%%%%%%%%%%%%%

    \section{Singularities}

    \paragraph{Isolated Singularities}
    A function has an \textit{isolated singularity} at \(z_0\)
    in \(\mathbb{C}\) if \(f\) is holomorphic in the punctured ball 
    \(B^\circ (z_0, R)\) for some \(R\in \mathbb{R}^+\) and, \(f\) is 
    not differentiable at \(z_0\).

    \paragraph{Types of Singularities and Zeroes}
    If \(c_n = 0\) for all \(n\) in \( \mathbb{Z}\), then, \(f(z_0) = 0.\)
    Otherwise, following examples are mutually exclusive and exhaustive

    \begin{itemize}
        \item There are infinitely many \(n\in \mathbb{Z}^-\) such that \(c_n \neq 0\).
        Then, \(z_0\) is an \textbf{essential} singularity.
        \item There are no \(n\in \mathbb{Z}\) such that \(c_n \neq 0\).
        Then, \(z_0\) is a removable singularity. Otherwise, if there exist
        \(m\in \mathbb{Z}\) such that \(c_m \neq 0\) and \(c_n = 0\)
        for all \(n < m\) then, \(f\) has a zero of order \(m\) at \(z_0\).
        \item There is atleast one, but finitely many \(n\in \mathbb{Z}\)
        such that \(c_n \neq 0\). Then there exists \(m\in \mathbb{Z}^-\)
        such that \(c_m \neq 0\) and \(c_n = 0\) for all \(n < m\).
        Then, \(f\) has a pole of order \(-M \)
    \end{itemize}

%%%%%%%%%%%%%%%%%%%%%%%%%%%%%%%%%%%%%%%%%%%%%%%%%%%%%%%%%%%%%%%%%%%%%%%%%%
%%%%%%%%%%%%%%%%%%%%%%%%%%%%%%%%%%%%%%%%%%%%%%%%%%%%%%%%%%%%%%%%%%%%%%%%%%
%%%%%%%%%%%%%%%%%%%%%%%%%%%%%%%%%%%%%%%%%%%%%%%%%%%%%%%%%%%%%%%%%%%%%%%%%%

    \section{Residues and Residue Theorem}
    \subsection{Residues}
    
    \paragraph{Definition}
    Suppose that the function \(f\) has an isolated singularity at \(z_0.\)
    Then, \(f\) is holomorphic on some punctured ball \(B^\circ z_0\)
    and, has a Laurent series there of form 
    \(\sum_{n=-\infty}^{\infty} c_n (z-z_0)^n\). The residue of \(f\) at \(z_0\)
    is written as:
    \[\mathrm{Res}(f, z_0) = c_{-1}. \]

    \paragraph{Laurent's Theorem Connection}
    From Laurent's theorem, if \(f\) is holomorphic in the area
    \(B^\circ (z_0, 0)\)
    then, 
    \[ \mathrm{Res}(f, z_0) = \frac{1}{2\pi i} \int_\Gamma f(z)dz.\]

    
    \subsection{Cauchy's Residue Theorem}
    \paragraph{Residue Theorem}
    Suppose that \(\Gamma\) is a simple, closed contour with the standard
    orientation in a domain \(\Omega\) and \(f\in H(\Omega)\) and,
    that \(\mathrm{Int}(\Gamma) \cap \Omega
    = \mathrm{Int}(\Gamma) \backslash \{z_1, z_2, z_3, \dots, z_K \}.\)
    Then,
    \[ \int_\Gamma f(z)dz = 2\pi i \sum_{k=1}^{K} \mathrm{Res}(f, z_k).\]
    
    \subsection{Computing Singularities}
    Suppose that \(f\) has an isolated singularity at the point \(z_0\).
    The, there are a few ways to computer the residue at \(z_0\), depending
    on the type of singularity.

    \paragraph{Removable Singularity}
    The residue of removable singularities is zero.
    \paragraph{Essential Singularities}
    There is no formulaic method to compute residues at essential
    singularities. A possible approach may be to consider integration, or
    manipulation of series.
    \paragraph{Poles of non-zero order}
    Suppose that \(f\) has a pole of order \(N\) at \(z_0\). Then,
    \[ \mathrm{Res}(f, z_0) = \frac{1}{(N-1)!} 
    \lim_{z\to z_0} \frac{d^{N-1}}{dz^{N-1}} (z-z_0)^N f(z).\]

    \paragraph{The p/q' formula}
    Suppose that \(f(z) = \frac{p(z)}{q(z)}\) in \(\Omega\) and that 
    \(p(z) \neq 0,\) while \(q(z) = 0\). If \(z_0\) is a simple pole of
    order \(1\) then,
    \[ \mathrm{Res}(f, z_0) = \frac{p(z_0)}{q'(z_0)}.\]
    
    

\end{document}

